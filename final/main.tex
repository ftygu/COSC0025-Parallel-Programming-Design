% -*- coding: utf-8 -*-
%-------------------------designed by zcf--------------
\documentclass[UTF8,a4paper,10pt]{ctexart}
\usepackage[left=3.17cm, right=3.17cm, top=2.74cm, bottom=2.74cm]{geometry}
\usepackage{amsmath}
\usepackage{graphicx,subfig}
\usepackage{float}
\usepackage{cite}
\usepackage{caption}
\usepackage{enumerate}
\usepackage{booktabs} %表格
\usepackage{multirow}
\newcommand{\tabincell}[2]{\begin{tabular}{@{}#1@{}}#2\end{tabular}}  %表格强制换行
%-------------------------字体设置--------------
% \usepackage{times} 
\usepackage{ctex}
\setCJKmainfont[ItalicFont=Noto Sans CJK SC Bold, BoldFont=Noto Serif CJK SC Black]{Noto Serif CJK SC}
\newcommand{\yihao}{\fontsize{26pt}{36pt}\selectfont}           % 一号, 1.4 倍行距
\newcommand{\erhao}{\fontsize{22pt}{28pt}\selectfont}          % 二号, 1.25倍行距
\newcommand{\xiaoer}{\fontsize{18pt}{18pt}\selectfont}          % 小二, 单倍行距
\newcommand{\sanhao}{\fontsize{16pt}{24pt}\selectfont}  %三号字
\newcommand{\xiaosan}{\fontsize{15pt}{22pt}\selectfont}        % 小三, 1.5倍行距
\newcommand{\sihao}{\fontsize{14pt}{21pt}\selectfont}            % 四号, 1.5 倍行距
\newcommand{\banxiaosi}{\fontsize{13pt}{19.5pt}\selectfont}    % 半小四, 1.5倍行距
\newcommand{\xiaosi}{\fontsize{12pt}{18pt}\selectfont}            % 小四, 1.5倍行距
\newcommand{\dawuhao}{\fontsize{11pt}{11pt}\selectfont}       % 大五号, 单倍行距
\newcommand{\wuhao}{\fontsize{10.5pt}{15.75pt}\selectfont}    % 五号, 单倍行距
%-------------------------章节名----------------
\usepackage{ctexcap} 
\CTEXsetup[name={,、},number={ \chinese{section}}]{section}
\CTEXsetup[name={(,)},number={\chinese{subsection}}]{subsection}
\CTEXsetup[name={,.},number={\arabic{subsubsection}}]{subsubsection}
%-------------------------页眉页脚--------------
\usepackage{fancyhdr}
\pagestyle{fancy}
\lhead{\kaishu \leftmark}
% \chead{}
\rhead{\kaishu 并行程序设计实验报告}%加粗\bfseries 
\lfoot{}
\cfoot{\thepage}
\rfoot{}
\renewcommand{\headrulewidth}{0.1pt}  
\renewcommand{\footrulewidth}{0pt}%去掉横线
\newcommand{\HRule}{\rule{\linewidth}{0.5mm}}%标题横线
\newcommand{\HRulegrossa}{\rule{\linewidth}{1.2mm}}
%-----------------------伪代码------------------
\usepackage{algorithm}  
\usepackage{algorithmicx}  
\usepackage{algpseudocode}  
\floatname{algorithm}{Algorithm}  
\renewcommand{\algorithmicrequire}{\textbf{Input:}}  
\renewcommand{\algorithmicensure}{\textbf{Output:}} 
\usepackage{lipsum}  
\makeatletter
\newenvironment{breakablealgorithm}
  {% \begin{breakablealgorithm}
  \begin{center}
     \refstepcounter{algorithm}% New algorithm
     \hrule height.8pt depth0pt \kern2pt% \@fs@pre for \@fs@ruled
     \renewcommand{\caption}[2][\relax]{% Make a new \caption
      {\raggedright\textbf{\ALG@name~\thealgorithm} ##2\par}%
      \ifx\relax##1\relax % #1 is \relax
         \addcontentsline{loa}{algorithm}{\protect\numberline{\thealgorithm}##2}%
      \else % #1 is not \relax
         \addcontentsline{loa}{algorithm}{\protect\numberline{\thealgorithm}##1}%
      \fi
      \kern2pt\hrule\kern2pt
     }
  }{% \end{breakablealgorithm}
     \kern2pt\hrule\relax% \@fs@post for \@fs@ruled
  \end{center}
  }
\makeatother
%------------------------代码-------------------
\usepackage{xcolor} 
\usepackage{listings} 
\lstset{ 
breaklines,%自动换行
basicstyle=\small,
escapeinside=``,
keywordstyle=\color{ blue!70} \bfseries,
commentstyle=\color{red!50!green!50!blue!50},% 
stringstyle=\ttfamily,% 
extendedchars=false,% 
linewidth=\textwidth,% 
numbers=left,% 
numberstyle=\tiny \color{blue!50},% 
frame=trbl% 
rulesepcolor= \color{ red!20!green!20!blue!20} 
}
%------------超链接----------
\usepackage[colorlinks,linkcolor=black,anchorcolor=blue]{hyperref}
%------------------------TODO-------------------
\usepackage{enumitem,amssymb}
\newlist{todolist}{itemize}{2}
\setlist[todolist]{label=$\square$}
% for check symbol 
\usepackage{pifont}
\newcommand{\cmark}{\ding{51}}%
\newcommand{\xmark}{\ding{55}}%
\newcommand{\done}{\rlap{$\square$}{\raisebox{2pt}{\large\hspace{1pt}\cmark}}\hspace{-2.5pt}}
\newcommand{\wontfix}{\rlap{$\square$}{\large\hspace{1pt}\xmark}}
%------------------------水印-------------------
\usepackage{tikz}
\usepackage{xcolor}
\usepackage{eso-pic}

\newcommand{\watermark}[3]{\AddToShipoutPictureBG{
\parbox[b][\paperheight]{\paperwidth}{
\vfill%
\centering%
\tikz[remember picture, overlay]%
  \node [rotate = #1, scale = #2] at (current page.center)%
    {\textcolor{gray!80!cyan!30!magenta!30}{#3}};
\vfill}}}



%———————————————————————————————————————————正文———————————————————————————————————————————————
%----------------------------------------------
\begin{document}
\begin{titlepage}
    \begin{center}
    \includegraphics[width=0.8\textwidth]{NKU.png}\\[1cm]    
    \textsc{\Huge \kaishu{\textbf{南\ \ \ \ \ \ 开\ \ \ \ \ \ 大\ \ \ \ \ \ 学}} }\\[0.9cm]
    \textsc{\huge \kaishu{\textbf{计\ \ 算\ \ 机\ \ 学\ \ 院}}}\\[0.5cm]
    \textsc{\Large \textbf{并行程序设计实验报告}}\\[0.8cm]
    \HRule \\[0.9cm]
    { \LARGE \bfseries 期末研究开题报告}\\[0.4cm]
    \HRule \\[2.0cm]
    \centering
    \textsc{\LARGE 张明昆 2211585\kaishu{\ \ \ \ }}\\[0.5cm]
    \textsc{\LARGE \kaishu{年级\ :\ 2022级}}\\[0.5cm]
    \textsc{\LARGE \kaishu{专业\ :\ 计算机科学与技术}}\\[0.5cm]
    \textsc{\LARGE \kaishu{指导教师\ :\ 王刚}}\\[0.5cm]
    \vfill
    {\Large \today}
    \end{center}
\end{titlepage}
%-------------摘------要--------------
\newpage
\thispagestyle{empty}
\renewcommand{\abstractname}{\kaishu \sihao \textbf{摘要}}
    \begin{abstract}
本开题报告旨在探索Gröbner基高斯消元法的并行优化,以解决多元多项式理想成员判定问题中的高计算复杂度。Gröbner基作为数学和计算代数领域的一个重要工具,虽然在多个领域有广泛应用,但其处理大规模问题时的效率低下是一个显著的短板。本研究提出的预计探索方向和实验设计,旨在通过并行计算和算法优化,显著提升Gröbner基高斯消元法在处理复杂问题时的性能。
        \noindent  %顶格
        \textbf{\\\ 关键字:Parallel Gröbner基高斯消元法}\textbf{} \\\ \\\
    \end{abstract}
%----------------------------------------------------------------
\tableofcontents
%----------------------------------------------------------------
\newpage
\watermark{60}{10}{NKU}
\setcounter{page}{1}
\section{背景介绍}
\subsection{Gröbner基简史}
Gröbner基,这个数学界的“瑞士军刀”,是经过数十年演变而来的。早在1927年,F.S.Macauly就引入了全序的概念,对多项式的单项式进行排序,这为Gröbner基的诞生奠定了基础。而后,在1964年,H.Hironaka在研究奇点分解时,引入了多项式除法算法,进一步拓宽了这一领域的研究。紧接着,在1965年,奥地利数学家B.Buchberger不仅引入了项序,确保了多项式除法的结果唯一性,还提出了S-多项式的概念,并设计了计算Gröbner基的算法,即著名的Buchberger算法,从而开辟了解决多元多项式理想问题的新途径。
\subsection{Gröbner基的应用}
在密码学中,Gröbner基扮演了解密的钥匙角色,特别是在处理多变量多项式方程系统的问题上,这些方程系统是多变量公钥密码体系的核心。通过解这些方程组,可以揭露加密信息,仿冒数字签名。然而,尽管Gröbner基提供了一种有效的求解工具,它的计算复杂度极高,早期的Buchberger算法效率低下,因此研究者们开发了更高效的算法,如F4和F5。

Gröbner基的计算虽然面临NP难题的挑战,但通过探索和利用系统内部的结构特征,可以显著提高算法的效率。例如,在某些布尔Gröbner基计算中,通过特殊的高斯消元方法来加速,展示了并行化算法潜力的一面。这正是我们研究的焦点,旨在通过并行计算,解锁Gröbner基算法的新可能。
\section{前人研究}
在数学和计算代数领域,Gröbner基的研究和开发已经走过了一段漫长而富有成果的道路。自Buchberger首次提出Gröbner基求解算法以来,众多科学家通过引入新的概念、方法和优化技术,极大地推动了这一领域的发展。以下是对这一进程的总结,重点介绍了几个关键的里程碑和最新进展。
\begin{itemize}
  \item Buchberger首次提出了Gröbner基求解算法,为后续研究奠定了基础。这一算法虽然开创性,但在处理复杂问题时效率不高。
  \item 继Buchberger之后,Lazard引入了线性代数方法到Gröbner基的求解中,这一方法为后续的算法优化提供了新的思路。
  \item Faugère提出了基于线性代数的F4算法和基于标签的F5算法,这两个算法被认为是目前求解Gröbner基最高效的方法。尤其是F5算法,它通过避免冗余计算和引入额外的检测标准来减少无效计算,虽然在执行效率上仍有提升空间。
  \item 在F4和F5算法的基础上,诸如Arri-Perry(AP)\cite{2}、Gao-Guan-Volny(G2V)\cite{4}和Gao-Volny-Wang(GVW)\cite{5}等并行算法应运而生,进一步提高了Gröbner基计算的效率。
除此之外,研究者还探索了对Macaulay矩阵的并行化改造、高效的约化准则以及利用MPI技术处理算法中间项的约化部分等优化方法\cite{7}
  \item Rodrigo Alexander Castro Campos等人提出了针对布尔Gröbner基的有效实现策略,包括对F4算法的变体、新的并行实现Buchberger准则以及待定多项式的改进处理方法。\cite{3}
\end{itemize}

通过这些年的不懈努力和创新,Gröbner基的求解算法已经从早期的Buchberger算法发展到今天的高效并行算法。
\section{问题简介}
\subsection{Gröbner基的挑战}
Gröbner基为非线性多项式理想的成员判定提供了一种强大的计算代数方法,但其计算复杂度高,尤其在涉及大型问题时,传统算法往往难以承受计算负担。
\subsection{特殊高斯消元算法}
为了优化Gröbner基的计算过程,引入了一种特殊的高斯消元方法。这种方法针对的是有限域GF(2)上的运算,其特点包括:
\begin{itemize}
  \item 运算数与规则:仅涉及0和1的运算,其中加法和乘法遵循特定的简化规则。
  \item 消元子模式:矩阵分为“消元子”和“被消元行”,其中消元子负责减去操作,被消元行则在消去过程中可能转换为消元子。
\end{itemize}
\subsection{高效实现的追求}
我们的研究旨在探索如何通过并行计算和算法优化,提高这一特殊高斯消元算法的效率,进而为Gröbner基的计算提供高效支持。这包括但不限于:
\begin{itemize}
  \item 数据批处理:考虑矩阵规模可能达到百万级,如何有效地分批处理数据以适应内存限制。
  \item 并行化策略:如何利用并行计算优化消元子与被消元行的处理,加速整个消去过程。
  \item 优化消元过程:探索算法内部如何减少冗余计算,提高消元效率,包括消元行的升格机制和空行处理。
\end{itemize}
\section{拟采取的研究方案}
在本研究中,我们将探索高斯消元法的并行化处理方法,特别关注于Gröbner基的计算过程,旨在提高高斯消元在多元多项式理想成员判定问题中的计算效率。高斯消元过程的并行化,特别是在处理大规模数据集时,具有显著的性能提升潜力。本研究将通过一系列实验,系统地探索和评估不同并行化策略在高斯消元法中的应用效果。

实验将在包括Intel Devcloud、鲲鹏、金山云等多个平台上进行,利用MPI、OpenMP、SIMD等技术进行并行化优化,同时使用CUDA和OneAPI探索GPU优化的可能性。实验结果将基于性能提升比(Speedup)和执行时间等关键指标进行评估和比较。
\subsection{实验一:MPI多线程优化}
\begin{itemize}
  \item 目标:采用消息传递接口(MPI)实现高斯消元算法的多线程并行优化,并分析在不同任务规模下的性能表现。
  \item 方法:实验将以不同大小的数据集为基准,评估MPI并行化对算法性能的影响。
\end{itemize}
\subsection{实验二:任务划分策略}
\begin{itemize}
  \item 目标:比较循环划分和块划分两种不同的任务划分策略,在高斯消元并行处理中的性能表现。
  \item 方法:通过实验评估在相同条件下,不同任务划分方式对性能的具体影响。
\end{itemize}
\subsection{实验三:消息发送方式比较}
\begin{itemize}
  \item 目标:对比广播和流水线两种MPI消息发送方式,在并行高斯消元过程中的性能差异。
  \item 方法:实施两种通信机制,分析其在不同并行规模下的性能和效率。
\end{itemize}
\subsection{实验四:结合MPI与SIMD向量化}
\begin{itemize}
  \item 目标:使用MPI配合SIMD向量化处理,探究并行化和向量化结合对性能的影响。
  \item 方法:在MPI并行框架下,引入SIMD向量化指令,评估其对算法加速的贡献。
\end{itemize}
\subsection{实验五:多线程优化(MPI + OMP)}
\begin{itemize}
  \item 目标:结合MPI和OpenMP(OMP)进行多线程优化,分析其性能提升。
  \item 方法:在MPI并行化基础上,进一步通过OMP实现线程级并行,对比单纯MPI并行的性能提升。
\end{itemize}
\subsection{实验六:综合优化(MPI + SIMD + OMP)}
\begin{itemize}
  \item 目标:探索MPI、SIMD和OMP的综合并行优化效果。
  \item 方法:将MPI多进程并行、OMP多线程优化和SIMD向量化处理结合起来,评估综合优化策略的性能。
\end{itemize}
\subsection{实验七:平台迁移性能分析}
\begin{itemize}
  \item 目标:将openmp优化方法从x86平台迁移到arm平台上,分析性能表现。
  \item 方法:在两种硬件平台上实施相同的优化策略,比较并分析性能差异。
\end{itemize}
\subsection{实验八:进程数量对性能的影响}
\begin{itemize}
  \item 目标:研究不同进程数量对MPI并行性能的影响。
  \item 方法:在固定的线程数下,调整进程数量,记录并分析性能变化。
\end{itemize}

本研究预期通过上述实验设计,深入理解并行化策略在高斯消元法中的应用效果,特别是在处理Gröbner基计算问题时的性能提升潜力。我们希望找到一种或多种有效的并行化方法,能够显著提高算法的执行效率,特别是在处理大规模数据集时。
\newpage
\bibliographystyle{plain}
\bibliography{reference} 
\end{document}
